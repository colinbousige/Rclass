\documentclass[]{article}
\usepackage{lmodern}
\usepackage{amssymb,amsmath}
\usepackage{ifxetex,ifluatex}
\usepackage{fixltx2e} % provides \textsubscript
\ifnum 0\ifxetex 1\fi\ifluatex 1\fi=0 % if pdftex
  \usepackage[T1]{fontenc}
  \usepackage[utf8]{inputenc}
\else % if luatex or xelatex
  \ifxetex
    \usepackage{mathspec}
  \else
    \usepackage{fontspec}
  \fi
  \defaultfontfeatures{Ligatures=TeX,Scale=MatchLowercase}
\fi
% use upquote if available, for straight quotes in verbatim environments
\IfFileExists{upquote.sty}{\usepackage{upquote}}{}
% use microtype if available
\IfFileExists{microtype.sty}{%
\usepackage{microtype}
\UseMicrotypeSet[protrusion]{basicmath} % disable protrusion for tt fonts
}{}
\usepackage[margin=1in]{geometry}
\usepackage{hyperref}
\hypersetup{unicode=true,
            pdftitle={Exam - Reproducible data treatment with R},
            pdfborder={0 0 0},
            breaklinks=true}
\urlstyle{same}  % don't use monospace font for urls
\usepackage{graphicx,grffile}
\makeatletter
\def\maxwidth{\ifdim\Gin@nat@width>\linewidth\linewidth\else\Gin@nat@width\fi}
\def\maxheight{\ifdim\Gin@nat@height>\textheight\textheight\else\Gin@nat@height\fi}
\makeatother
% Scale images if necessary, so that they will not overflow the page
% margins by default, and it is still possible to overwrite the defaults
% using explicit options in \includegraphics[width, height, ...]{}
\setkeys{Gin}{width=\maxwidth,height=\maxheight,keepaspectratio}
\IfFileExists{parskip.sty}{%
\usepackage{parskip}
}{% else
\setlength{\parindent}{0pt}
\setlength{\parskip}{6pt plus 2pt minus 1pt}
}
\setlength{\emergencystretch}{3em}  % prevent overfull lines
\providecommand{\tightlist}{%
  \setlength{\itemsep}{0pt}\setlength{\parskip}{0pt}}
\setcounter{secnumdepth}{0}
% Redefines (sub)paragraphs to behave more like sections
\ifx\paragraph\undefined\else
\let\oldparagraph\paragraph
\renewcommand{\paragraph}[1]{\oldparagraph{#1}\mbox{}}
\fi
\ifx\subparagraph\undefined\else
\let\oldsubparagraph\subparagraph
\renewcommand{\subparagraph}[1]{\oldsubparagraph{#1}\mbox{}}
\fi

%%% Use protect on footnotes to avoid problems with footnotes in titles
\let\rmarkdownfootnote\footnote%
\def\footnote{\protect\rmarkdownfootnote}

%%% Change title format to be more compact
\usepackage{titling}

% Create subtitle command for use in maketitle
\providecommand{\subtitle}[1]{
  \posttitle{
    \begin{center}\large#1\end{center}
    }
}

\setlength{\droptitle}{-2em}

  \title{Exam - Reproducible data treatment with R}
    \pretitle{\vspace{\droptitle}\centering\huge}
  \posttitle{\par}
    \author{}
    \preauthor{}\postauthor{}
      \predate{\centering\large\emph}
  \postdate{\par}
    \date{2019/12/03}


\begin{document}
\maketitle

\begin{center}\rule{0.5\linewidth}{\linethickness}\end{center}

Download the \href{http://lmi.cnrs.fr/r/Data/Exam.zip}{Exam.zip} file
and unzip it in a folder called ``\texttt{Rexam}''.

Open the \texttt{exam.Rmd} file and rename it as
\texttt{NAME\_FirstName.Rmd}. Put your name in the header of the file.

You will write all your text and code in this Rmd file, and you will
email
(\href{mailto:colin.bousige@univ-lyon1.fr}{\nolinkurl{colin.bousige@univ-lyon1.fr}})
to me the file at the end of the exam. \textbf{The file should compile
(knit) with no error}. Unless specifically specified, all graphs can be
done either using base graphics or \texttt{ggplot2}.

\textbf{Internet access and research is \emph{authorized}}.

\textbf{Sharing your answers through email, Facebook or any other mean
\emph{is not}}.

\begin{center}\rule{0.5\linewidth}{\linethickness}\end{center}

\hypertarget{exercise-1-4-points}{%
\section{Exercise 1 (4 points)}\label{exercise-1-4-points}}

\begin{enumerate}
\def\labelenumi{\arabic{enumi}.}
\tightlist
\item
  Print the 6 first lines of the R-built-in data.frame \texttt{trees}
\item
  Print only the column names
\item
  What is the dimension of \texttt{trees}?
\item
  Plot the trees height and volume as a function of their girth in two
  different graphs. Make sure the axis labels are clear
\item
  In each graph, add a red dashed line corresponding to the relevant
  correlation that you observe (average value, linear
  correlation\ldots{})
\item
  Explain your choice and write the corresponding values (average value
  and standard deviation, or slope, intercept and corresponding errors).
  Round the values to 2 decimals.
\end{enumerate}

\begin{center}\rule{0.5\linewidth}{\linethickness}\end{center}

\hypertarget{exercise-2-6-points}{%
\section{Exercise 2 (6 points)}\label{exercise-2-6-points}}

\begin{enumerate}
\def\labelenumi{\arabic{enumi}.}
\tightlist
\item
  Print the 3 first lines of the R-built-in data.frame
  \texttt{USArrests}. This data set contains statistics about violent
  crime rates by US state. The numbers are given per 100 000
  inhabitants, except for \texttt{UrbanPop} which is a percentage.
\item
  What is the average murder rate in the whole country?
\item
  What is the state with the highest assault rate?
\item
  Create a subset of \texttt{USArrests} gathering the data for states
  with an urban population above (including) 80\%.
\item
  How many states does that correspond to?
\item
  Within these states, what is the state with the smallest rape rate?
\item
  Print this subset ordered by decreasing urban population.
\item
  Print this subset ordered by decreasing urban population and
  increasing murder rate.
\item
  Plot an histogram of the percentage of urban population with a binning
  of 5\%. Add a vertical red line marking the average value. Make sure
  the x axis shows the {[}0,100{]} range.
\item
  Is there a correlation between the percentage of urban population and
  the various violent crime rates? argument your answer with plots.
\end{enumerate}

\begin{center}\rule{0.5\linewidth}{\linethickness}\end{center}

\hypertarget{exercise-3-10-points}{%
\section{Exercise 3 (10 points)}\label{exercise-3-10-points}}

In high-pressure experiments, the pressure in the Diamond Anvil Cell
(DAC) is calibrated through the measure of the Raman shift of a tiny
ruby crystal placed in the pressure transmitting medium next to the
measured sample.

\begin{enumerate}
\def\labelenumi{\arabic{enumi}.}
\item
  Write a function returning the pressure \(P\) as a function of the
  ruby Raman shift position \(\omega\) and the excitation laser
  wavelength \(\lambda_l\):
  \[P(\omega, \lambda_l) = \frac{A}{B}\left[\left(\frac{\lambda}{\lambda_0}\right)^B-1\right] (GPa)\]
  where \(A=1876\) and \(B=10.71\), \(\lambda\) is the the measured
  wavelength of the ruby \(R_1\) line (the most energetic one) and
  \(\lambda_0=694.24\) nm is the zero-pressure value at 298 K {[}1{]}.
  The relationship between the wavenumber \(\nu\) in cm\(^{-1}\) and the
  wavelength \(\lambda\) in nm is given by
  \(\nu(\text{cm}^{-1})=\frac{10^7}{\lambda(\text{nm})}\), and the Raman
  shift
  \(\omega=\Delta\nu=\nu_l-\nu=\frac{10^7}{\lambda_l}-\frac{10^7}{\lambda}\)
  (cm\(^{-1}\)).
\item
  Write a function returning a normalized Lorentzian as a function of
  its center \(x_0\) and its full width at half maximum \(\Gamma\):
  \[L(x)=\frac{\Gamma}{2\pi}\frac{1}{\frac{\Gamma^2}{4}+\left(x-x_0\right)^2}\]
\item
  Store the list of files containing \texttt{ruby} in their name in the
  \texttt{Data/} folder into a variable \texttt{flist}. Print its
  length.
\item
  Plot with points the first file in \texttt{flist}. Find the position
  of its maximum and store it in \texttt{xmax}. Guess roughly the
  parameters needed to fit the experimental data by
  \texttt{y0+A1*L(x,x1,FW1)+A2*L(x,x2,FW2)}, and add a blue line on the
  plot to represent this function.
\item
  Using \texttt{nls()}, fit the first spectrum in \texttt{flist} by
  \texttt{y0+A1*L(x,x1,FW1)+A2*L(x,x2,FW2)} and using the starting
  parameters you defined before. Plot the experimental data again and
  add the fitted spectrum as a red line.
\item
  Based on the above procedure, for each file in \texttt{flist} (so, use
  a \texttt{for} loop), fit the Raman spectrum by the sum of two
  Lorentzian functions, and store the fitting parameters into a
  data.frame called \texttt{ruby\_fit} also containing the names of the
  corresponding files. Attention: the initial guesses for amplitudes and
  widths can be constant, but the peaks positions should evolve for each
  spectrum. The difference between the two peaks is always roughly 30
  cm\(^{-1}\), and the largest peak is always the most energetic one.
  Check that your fits are correct by printing the experimental data and
  the fitted result at each iteration (add the name of the file as the
  plot title).
\item
  Add a column in \texttt{ruby\_fit} corresponding to the estimated
  pressure rounded to 1 decimal. The excitation wavelength in this
  experiment was 532 nm. Print the resulting \texttt{ruby\_fit} table
  using \texttt{knitr::kable(ruby\_fit)}
\item
  Store all file names containing ``RBM'' into a variable \texttt{fRBM}.
  Load all the corresponding spectra into a single \texttt{data.frame}
  called \texttt{spec} with 3 columns: Raman shift \(\omega\),
  Intensity, Pressure. Of course, the indexes in the file names between
  the ruby and RBM files match. In the Intensity column, store the
  intensity normalized to {[}0,1{]}.
\item
  Using \texttt{ggplot2}, plot with points the stacked normalized RBM
  band spectra vertically shifted by P, with a color for each spectrum
  corresponding to the pressure. Make the plot interactive.
\end{enumerate}

\hypertarget{references}{%
\section*{References}\label{references}}
\addcontentsline{toc}{section}{References}

\hypertarget{refs}{}
\leavevmode\hypertarget{ref-Chijioke2005}{}%
{[}1{]} Chijioke \emph{et al.} `The ruby pressure standard to 150 GPa'.
\emph{J Appl Phys} \textbf{98,} 114905 (2005). DOI:
\href{https://doi.org/10.1063/1.2135877}{10.1063/1.2135877}


\end{document}
